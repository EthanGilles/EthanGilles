%\documentclass[10pt,a4paper]{article}
\documentclass[landscape, 10pt,a4paper]{report}
% Packages
\usepackage{fancyhdr}           % For header and footer
\usepackage{multicol}           % Allows multicols in tables
\usepackage{tabularx}           % Intelligent column widths
\usepackage{tabulary}           % Used in header and footer
\usepackage{hhline}             % Border under tables
\usepackage{graphicx}           % For images
\usepackage{xcolor}             % For hex colours
%\usepackage[utf8x]{inputenc}    % For unicode character support
\usepackage[T1]{fontenc}        % Without this we get weird character replacements
\usepackage{colortbl}           % For coloured tables
\usepackage{setspace}           % For line height
\usepackage{lastpage}           % Needed for total page number
\usepackage{seqsplit}           % Splits long words.
%\usepackage{opensans}          % Can't make this work so far. Shame. Would be lovely.
\usepackage[normalem]{ulem}     % For underlining links
% Most of the following are not required for the majority
% of cheat sheets but are needed for some symbol support.
\usepackage{amsmath}            % Symbols
\usepackage{MnSymbol}           % Symbols
\usepackage{wasysym}            % Symbols
%\usepackage[english,german,french,spanish,italian]{babel}              % Languages

\usepackage[margin=1.65in]{geometry}

\usepackage{pdflscape}
\usepackage{adjustbox}

% Document Info
\author{nhatlong0605}
\pdfinfo{
  /Title (lfcs-module6-serviceconfiguration.pdf)
  /Creator (Cheatography)
  /Author (nhatlong0605)
  /Subject (Python Cheatsheet)
}

% Lengths and widths
\addtolength{\textwidth}{7cm}
\addtolength{\textheight}{3cm}  % -1
\addtolength{\hoffset}{-3.2cm}
\addtolength{\voffset}{-3cm}
\setlength{\tabcolsep}{0.3cm} % Space between columns og=0.2
\setlength{\headsep}{-12pt} % Reduce space between header and content
\setlength{\headheight}{85pt} % If less, LaTeX automatically increases it
\renewcommand{\footrulewidth}{0pt} % Remove footer line
\renewcommand{\headrulewidth}{0pt} % Remove header line
\renewcommand{\seqinsert}{\ifmmode\allowbreak\else\-\fi} % Hyphens in seqsplit
% This two commands together give roughly
% the right line height in the tables
\renewcommand{\arraystretch}{1.3}
\onehalfspacing

% Commands
\newcommand{\SetRowColor}[1]{\noalign{\gdef\RowColorName{#1}}\rowcolor{\RowColorName}} % Shortcut for row colour
\newcommand{\mymulticolumn}[3]{\multicolumn{#1}{>{\columncolor{\RowColorName}}#2}{#3}} % For coloured multi-cols
\newcolumntype{x}[1]{>{\raggedright}p{#1}} % New column types for ragged-right paragraph columns
\newcommand{\tn}{\tabularnewline} % Required as custom column type in use

% Font and Colours
\definecolor{LogoBack}{HTML}{ffffff}

\definecolor{HeadBackground}{HTML}{336699}
\definecolor{FootBackground}{HTML}{9fa1a4}
\definecolor{TextColor}{HTML}{231f20}
\definecolor{DarkBackground}{HTML}{163e70} % 1D41A3
\definecolor{LightBackground}{HTML}{D8DEEF}
\renewcommand{\familydefault}{\sfdefault}
\color{TextColor}





% Header and Footer
\pagestyle{fancy}
\fancyhead{} % Set header to blank
\fancyfoot{} % Set footer to blank
\fancyhead[L]{
\noindent
\begin{multicols}{3}
\begin{tabulary}{5.8cm}{C}
    \SetRowColor{LogoBack}
    \vspace{-7pt}
    {\parbox{\dimexpr\textwidth-2\fboxsep\relax}{\noindent
        \hspace*{-6pt}\includegraphics[width=5.8cm]{logo.jpg}\\Department of Computer Sciences}
    }
\end{tabulary}
\columnbreak
\begin{tabulary}{11cm}{L}
    \vspace{-2pt}\large{\bf{\textcolor{DarkBackground}{\textsf{Unix/Linux Commands}}}} \\
    \normalsize{ \textcolor{DarkBackground}{Department of Computer Sciences}}
\end{tabulary}
\columnbreak
\begin{tabulary}{11cm}{L}
    \vspace{-2pt}\large{\bf{\textcolor{DarkBackground}{\textsf{Unix/Linux Commands}}}} \\
    \normalsize{ \textcolor{DarkBackground}{Department of Computer Sciences}}
\end{tabulary}
\end{multicols}}






\fancyfoot[L]{ \footnotesize
\noindent
\begin{multicols}{3}
\begin{tabulary}{5.8cm}{LL}
  \SetRowColor{FootBackground}
  \mymulticolumn{2}{p{5.377cm}}{\bf\textcolor{white}{Programs}}  \\
  \vspace{-2pt}BS Computer Science \\
 MS Data Science\\
  \end{tabulary}
\vfill
\columnbreak
\begin{tabulary}{5.8cm}{L}
  \SetRowColor{FootBackground}
  \mymulticolumn{1}{p{5.377cm}}{\bf\textcolor{white}{Cheat Sheet}}  \\
   \vspace{-2pt}Published 27th September, 2018.\\
   Updated 28th September, 2018.\\
   Page {\thepage} of \pageref{LastPage}.
\end{tabulary}
\vfill
\columnbreak
\begin{tabulary}{5.8cm}{L}
  \SetRowColor{FootBackground}
  \mymulticolumn{1}{p{5.377cm}}{\bf\textcolor{white}{Contact}}  \\
  \SetRowColor{white}
  \vspace{-5pt}
  %\includegraphics[width=48px,height=48px]{dave.jpeg}
  Dr. James Quinlan\\
  Chair, Dept. of Computer Science\\
  (207) 780-4723\\
  james.quinlan@maine.edu
\end{tabulary}
\end{multicols}}








\begin{document}
\raggedright
\raggedcolumns
% \begin{landscape}
% Set font size to small. Switch to any value
% from this page to resize cheat sheet text:
% www.emerson.emory.edu/services/latex/latex_169.html
\footnotesize % Small font.

\begin{multicols*}{3}

\begin{tabularx}{8.6cm}{X}
\SetRowColor{DarkBackground}
\mymulticolumn{1}{x{8.6cm}}{\bf\textcolor{white}{General Syntax}}  \tn
\SetRowColor{white}
\mymulticolumn{1}{x{8.6cm}}{Command [Options] [Arguments] \newline % Row Count 1 (+ 1)
Command: The actual command you want to execute \newline % Row Count 
Options: Modifies the behavior of a command. Begins with a hyphen `-' or double hyphen `--' \newline % Row Count 4 (+ 1)
Arguments: The objects that the command operates on. Can be file names, directories, or other data.  \newline % Row Count 5 (+ 1)
} \tn 
\hhline{>{\arrayrulecolor{DarkBackground}}-}
\end{tabularx}


\par\addvspace{1.3em}

\begin{tabularx}{8.6cm}{x{3.44 cm} x{4.56 cm} }


\SetRowColor{DarkBackground}
\mymulticolumn{2}{x{8.6cm}}{\bf\textcolor{white}{Help/Info Commands}}  \tn
% Row 0
\SetRowColor{LightBackground}
help & provide information related to Shell \underline{built-in} commands \tn 
% Row Count 2 (+ 2)
% Row 1
\SetRowColor{white}
type & Provides the command type \tn 
% Row Count 4 (+ 2)
% Row 2
\SetRowColor{LightBackground}
whatis & A one-line description \tn 
% Row Count 6 (+ 2)
% Row 3
\SetRowColor{white}
man & Display the manual or `man pages' for a given command, plain text \tn 
% Row Count 8 (+ 2)

\SetRowColor{LightBackground}
info & In-depth document for a given command, hypertext \tn 

\SetRowColor{white}
apropos & Find a command's name \tn 

\SetRowColor{LightBackground}
which & Find the associated executable file for a command \tn 

\hhline{>{\arrayrulecolor{DarkBackground}}--}
\end{tabularx}


\par\addvspace{1.1em}

\begin{tabularx}{8.6cm}{X}
\SetRowColor{DarkBackground}
\mymulticolumn{1}{x{8.6cm}}{\bf\textcolor{white}{Redirection/Pipes}}  \tn
\SetRowColor{LightBackground}
\mymulticolumn{1}{x{8.6cm}}{
"<" - Input redirection \newline
Redirects the standard input of a command to a file. Command will read input from a file instead of the keyboard. \newline 
  ">" - Output redirection \newline 
Redirects the standard output of a command to a file. If the file already exists, it will be \underline{overwritten}. 
Otherwise, a new file is created. \newline
" | " Pipe - Create a chain of commands \newline 
Syntax: \textit{command1} | \textit{command2 } \newline
Allows to send the output of one command as input for another. 
Here, the output of command1 is the input for command2.
}
\tn 
\hhline{>{\arrayrulecolor{DarkBackground}}-}
\end{tabularx}

\par\addvspace{1.3em}


\begin{tabularx}{8.6cm}{x{3.44 cm} x{4.56 cm} }

\SetRowColor{DarkBackground}
\mymulticolumn{2}{x{8.6cm}}{\bf\textcolor{white}{File Management}}  \tn

\SetRowColor{LightBackground}
ls & List files and sub-directories \tn 

\SetRowColor{LightBackground}
cd \textit{dir} & Change directory to \textit{dir} \tn 

\SetRowColor{LightBackground}
pwd & Show current directory \tn 

\SetRowColor{white}
rm \textit{file} & Remove \textit{file} \tn 

\SetRowColor{LightBackground}
rm -r \textit{dir} & Recursively remove the directory \textit{dir} \tn 

\SetRowColor{white}
cp \textit{file1} \textit{file2} & Copy \textit{file1} to \textit{file2} \tn 

\SetRowColor{LightBackground}
cp -r \textit{dir1} \textit{dir2} & Recursively copy \textit{dir1} to \textit{dir2} \tn 
\hhline{>{\arrayrulecolor{DarkBackground}}--}
\end{tabularx}

\par\addvspace{1.1em}


\begin{tabularx}{8.6cm}{x{3.44 cm} x{4.56 cm} }

\SetRowColor{DarkBackground}
\mymulticolumn{2}{x{8.6cm}}{\bf\textcolor{white}{Git Commands}}  \tn
% Row 0
\SetRowColor{LightBackground}
git init & Initialize an existing directory as a git repository \tn 
% Row Count 2 (+ 2)
% Row 1
\SetRowColor{white}
git clone [url] & Download a full repository from the origin using a URL \tn 
% Row Count 4 (+ 2)
% Row 2
\SetRowColor{LightBackground}
git status & Shows the modified files in the working directory that are staged for commit \tn 
% Row Count 6 (+ 2)
% Row 3
\SetRowColor{white}
git add [file] & Adds a file to the next commit \tn 
% Row Count 8 (+ 2)

\SetRowColor{LightBackground}
git diff & Shows differences in files that have been changed but not staged \tn 

\SetRowColor{white}
git commit -m '[message]' & Commit the staged changes as a new commit snapshot \tn 

\SetRowColor{LightBackground}
git branch & List all the branches. A `*' will appear next to the active branch \tn 

\SetRowColor{white}
git checkout [branch] & Switch to another branch and bring it to your working directory \tn 

\SetRowColor{LightBackground}
git push [alias] [branch] & Transmit local branch commits to the remote repository \tn 

\SetRowColor{white}
git pull & Fetch and merge any commits from the remote repository \tn 

\hhline{>{\arrayrulecolor{DarkBackground}}--}
\end{tabularx}


\par\addvspace{1.1em}



\begin{tabularx}{8.6cm}{X}
\SetRowColor{DarkBackground}
\mymulticolumn{1}{x{8.6cm}}{\bf\textcolor{white}{Virtual Hosts}}  \tn
\SetRowColor{white}
\mymulticolumn{1}{x{8.6cm}}{- vim /etc/hosts; Create name resolution \newline % Row Count 1 (+ 1)
- cd /etc/httpd/conf.d/; vim account.example.com.conf : Create virtual host w/ content \newline % Row Count 3 (+ 2)
\textless{}VirtualHost *:80\textgreater{} \newline % Row Count 4 (+ 1)
ServerAdmin \seqsplit{webmaster@account.example.com} \newline % Row Count 5 (+ 1)
DocumentRoot /web/account \newline % Row Count 6 (+ 1)
ServerName account.example.com \newline % Row Count 7 (+ 1)
\textless{}/VirtualHost\textgreater{} \newline % Row Count 8 (+ 1)
- mkdir /web/account; cd /web/account; vim index.html : Create index file% Row Count 10 (+ 2)
} \tn 
\hhline{>{\arrayrulecolor{DarkBackground}}-}
\end{tabularx}




\par\addvspace{1.3em}










\begin{tabularx}{8.6cm}{x{3.52 cm} x{4.48 cm} }
\SetRowColor{DarkBackground}
\mymulticolumn{2}{x{8.6cm}}{\bf\textcolor{white}{FTP Server}}  \tn
% Row 0
\SetRowColor{LightBackground}
yum install vsftpd & Install Very Secure ftp \tn 
% Row Count 2 (+ 2)
% Row 1
\SetRowColor{white}
/etc/vsftpd & Main config file \tn 
% Row Count 3 (+ 1)
% Row 2
\SetRowColor{LightBackground}
yum install lftp & Install ftp client \tn 
% Row Count 4 (+ 1)
% Row 3
\SetRowColor{white}
lftp localhost & Connect to ftp \tn 
% Row Count 5 (+ 1)
\hhline{>{\arrayrulecolor{DarkBackground}}--}
\end{tabularx}





\par\addvspace{1.3em}




\begin{tabularx}{8.6cm}{x{3.52 cm} x{4.48 cm} }
\SetRowColor{DarkBackground}
\mymulticolumn{2}{x{8.6cm}}{\bf\textcolor{white}{FTP Server}}  \tn
% Row 0
\SetRowColor{LightBackground}
yum install vsftpd & Install Very Secure ftp \tn 
% Row Count 2 (+ 2)
% Row 1
\SetRowColor{white}
/etc/vsftpd & Main config file \tn 
% Row Count 3 (+ 1)
% Row 2
\SetRowColor{LightBackground}
yum install lftp & Install ftp client \tn 
% Row Count 4 (+ 1)
% Row 3
\SetRowColor{white}
lftp localhost & Connect to ftp \tn 
% Row Count 5 (+ 1)
\hhline{>{\arrayrulecolor{DarkBackground}}--}
\end{tabularx}





\par\addvspace{1.3em}




\begin{tabularx}{8.6cm}{x{3.52 cm} x{4.48 cm} }
\SetRowColor{DarkBackground}
\mymulticolumn{2}{x{8.6cm}}{\bf\textcolor{white}{FTP Server}}  \tn
% Row 0
\SetRowColor{LightBackground}
yum install vsftpd & Install Very Secure ftp \tn 
% Row Count 2 (+ 2)
% Row 1
\SetRowColor{white}
/etc/vsftpd & Main config file \tn 
% Row Count 3 (+ 1)
% Row 2
\SetRowColor{LightBackground}
yum install lftp & Install ftp client \tn 
% Row Count 4 (+ 1)
% Row 3
\SetRowColor{white}
lftp localhost & Connect to ftp \tn 
% Row Count 5 (+ 1)
\hhline{>{\arrayrulecolor{DarkBackground}}--}
\end{tabularx}





\par\addvspace{1.3em}




\begin{tabularx}{8.6cm}{x{3.52 cm} x{4.48 cm} }
\SetRowColor{DarkBackground}
\mymulticolumn{2}{x{8.6cm}}{\bf\textcolor{white}{FTP Server}}  \tn
% Row 0
\SetRowColor{LightBackground}
yum install vsftpd & Install Very Secure ftp \tn 
% Row Count 2 (+ 2)
% Row 1
\SetRowColor{white}
/etc/vsftpd & Main config file \tn 
% Row Count 3 (+ 1)
% Row 2
\SetRowColor{LightBackground}
yum install lftp & Install ftp client \tn 
% Row Count 4 (+ 1)
% Row 3
\SetRowColor{white}
lftp localhost & Connect to ftp \tn 
% Row Count 5 (+ 1)
\hhline{>{\arrayrulecolor{DarkBackground}}--}
\end{tabularx}





\par\addvspace{1.3em}





\begin{tabularx}{8.6cm}{X}
\SetRowColor{DarkBackground}
\mymulticolumn{1}{x{8.6cm}}{\bf\textcolor{white}{Configure Postfix}}  \tn
\SetRowColor{LightBackground}
\mymulticolumn{1}{x{8.6cm}}{\#Email delivery \newline vim /etc/postfix/main.cf \newline    relayhost = {[}192.168.4.200{]} \newline  \newline \#Email Receiving \newline vim /etc/postfix/main.cf \newline    myorigin = \$mydomain \newline    mynetworks = 192.168.4.0/24 \newline    inet\_protocols = ipv4} \tn 
\hhline{>{\arrayrulecolor{DarkBackground}}-}
\end{tabularx}






\par\addvspace{1.3em}







\begin{tabularx}{8.6cm}{X}
\SetRowColor{DarkBackground}
\mymulticolumn{1}{x{8.6cm}}{\bf\textcolor{white}{Squid Proxy}}  \tn
\SetRowColor{LightBackground}
\mymulticolumn{1}{x{8.6cm}}{yum install squid \newline vim /etc/squid/squid.conf \newline systemctl enable -{}-now squid \newline  \newline \#Configure browser \newline Preferences -\textgreater{} Advanced -\textgreater{} Network -\textgreater{} Settings -\textgreater{} Manual: 192.168.4.240 port 3128} \tn 
\hhline{>{\arrayrulecolor{DarkBackground}}-}
\end{tabularx}






\par\addvspace{1.3em}







\begin{tabularx}{8.6cm}{X}
\SetRowColor{DarkBackground}
\mymulticolumn{1}{x{8.6cm}}{\bf\textcolor{white}{KVM}}  \tn
\SetRowColor{LightBackground}
\mymulticolumn{1}{x{8.6cm}}{yum group install "Virtualization Host" \newline systemctl start libvirtd \newline yum install virt-manager \newline virt (Start Virtual manger) \newline  \newline virsh list (Show available VMs) \newline virsh list -{}-all (Show all VMs) \newline virsh start {\emph{VM name}} \newline virsh stop {\emph{VM name}} \newline virsh autostart {\emph{VM name}} (Autostart after the host reboot)} \tn 
\hhline{>{\arrayrulecolor{DarkBackground}}-}
\end{tabularx}







\par\addvspace{1.3em}









\begin{tabularx}{8.6cm}{X}
\SetRowColor{DarkBackground}
\mymulticolumn{1}{x{8.6cm}}{\bf\textcolor{white}{Cache-only DNS Server}}  \tn
\SetRowColor{LightBackground}
\mymulticolumn{1}{x{8.6cm}}{yum install unbound \newline systemctl enable -{}-now unbound \newline iptables -A INPUT -p udp/tcp -dport 53 -j ACCEPT \newline vim \seqsplit{/etc/unbound/unbound.conf} \newline   \# interface: 0.0.0.0 \newline   \# access-control: 192.168.0.0/16 allow \newline   \#domain-insecure: "example.com" \newline   \# forward-zone: \newline                   name: "." \newline                   forward-addr: 8.8.8.8 \newline systemctl restart unbound \newline netstat -tulpen (Show listening port)} \tn 
\hhline{>{\arrayrulecolor{DarkBackground}}-}
\end{tabularx}







\par\addvspace{1.3em}










\begin{tabularx}{8.6cm}{x{4.8 cm} x{3.2 cm} }
\SetRowColor{DarkBackground}
\mymulticolumn{2}{x{8.6cm}}{\bf\textcolor{white}{Database}}  \tn
% Row 0
\SetRowColor{LightBackground}
yum install mariadb & Install Mariadb \tn 
% Row Count 1 (+ 1)
% Row 1
\SetRowColor{white}
\seqsplit{mysql\_secure\_installation} & Interactive setup for Mysql \tn 
% Row Count 3 (+ 2)
% Row 2
\SetRowColor{LightBackground}
mysql -u root -p & Login mysql using root \tn 
% Row Count 5 (+ 2)
% Row 3
\SetRowColor{white}
create database {\emph{dbname}} & Create database \tn 
% Row Count 6 (+ 1)
% Row 4
\SetRowColor{LightBackground}
use people; create table users(firstname varchar(20),....); & Create table \tn 
% Row Count 9 (+ 3)
% Row 5
\SetRowColor{white}
insert into user(...) values(...); & Insert values \tn 
% Row Count 11 (+ 2)
% Row 6
\SetRowColor{LightBackground}
select * from users & Show table \tn 
% Row Count 12 (+ 1)
\hhline{>{\arrayrulecolor{DarkBackground}}--}
\end{tabularx}










\par\addvspace{1.3em}









\begin{tabularx}{8.6cm}{X}
\SetRowColor{DarkBackground}
\mymulticolumn{1}{x{8.6cm}}{\bf\textcolor{white}{Configure NFS Server}}  \tn
\SetRowColor{LightBackground}
\mymulticolumn{1}{x{8.6cm}}{mkdir /nfsshare (Create folder for NFS) \newline vim /etc/exports \newline   /share *(rw,no\_root\_squash) \newline systemctl enable -{}-now nfs-server \newline showmount -e localhost (Check NFS server) \newline  \newline \#Mounting NFS Share persistently \newline mkdir -p /centos/nfs (Create mount point) \newline mount centos:/share /centos/nfs (Mount) \newline vim /etc/fstab \newline   \#centos:/share  /centos/nfs  nfs  \_netdev  0 0\textbackslash{} \newline umount /centos/nfs \newline mount -a} \tn 
\hhline{>{\arrayrulecolor{DarkBackground}}-}
\end{tabularx}








\par\addvspace{1.3em}










\begin{tabularx}{8.6cm}{X}
\SetRowColor{DarkBackground}
\mymulticolumn{1}{x{8.6cm}}{\bf\textcolor{white}{Configure Samba Server}}  \tn
\SetRowColor{LightBackground}
\mymulticolumn{1}{x{8.6cm}}{yum install samba \newline mv \seqsplit{/etc/samb/smb.conf.example} /etc/samba/smb.conf (Overwrite) \newline vim /etc/samba/smb.conf (Create a samba share) \newline   {[}share{]} \newline             comment = samba share \newline             path = /samba \newline             public = yes \newline mkdir /samba \newline semanage fcontext -a -t samba\_share\_t /samba \newline restorecon -Rv /samba \newline systemctl start smb \newline  \newline \#Create samba user \newline smbpasswd -a anna \newline  \newline \#smb client \newline yum install smbclient \newline smbclient -L localhost \newline  \newline \#Mount smb persistently \newline mount -o username=anna //centos/share /mnt \newline yum install clifs-utils (Install additonal utils) \newline mkdfir /centos/samba \newline vim /etc/fstab \newline   //centos/samba /centos/samba cifs  \newline    \_netdev,username=anna,password=password 0 0} \tn 
\hhline{>{\arrayrulecolor{DarkBackground}}-}
\end{tabularx}
\par\addvspace{1.3em}


% That's all folks
\end{multicols*}
 % \end{landscape}
\end{document}
